\section{Convergencia}
Ya hemos declarado informalmente que la distribución estacionaria describe el
comportamiento a largo plazo de la cadena, en el sentido de que si ejecutamos la cadena por un largo tiempo,
la distribución marginal de $X_{}$ converge a la distribución estacionaria $S$. El siguiente
el teorema establece que esto es cierto siempre que la cadena sea tanto irreductible como aperiódica.
Entonces, independientemente de las condiciones iniciales de la cadena, el PMF de Xn convergerá a la distribución estacionaria cuando  $n \longrightarrow \infty.$ 
theorema 11.3.6(Convergencia a la distribucion estacionaria). Sea $X_0,X_1,\dots $ sea una cadena de Markov con distribucion estacionaria $S$ y matrix de transicion $Q$, tal que la potencia $Q^{n}$ tiene todas las entradas positivas. Entonces $P(X_{n}=i)$ converge a $S_{i}$ mientras  $n \longrightarrow \infty.$. En terminos de la matriz de transicion, $Q^{n}$ converge a una matriz en la que cada fila es $S$.\\


\[
\left( \begin{array}{cccc}
1/3 & 2/3 \\
1/2 & 1/2 \end{array}  \right)^{n}
\rightarrow
\left( \begin{array}{cccc}
3/7 & 4/7 \\
3/7 & 4/7 \end{array}  \right)
cuando \phantom{x} n \rightarrow \infty
\]
\section{Reversibilidad}
Hemos visto que la distribución estacionaria de una cadena de Markov es extremadamente util para entender su comportamiento a largo plazo. Desafortunamente en general puede ser computacionalmente dificil de encontrar la distribución estacionaria cuando el espacio de estados es grande.\\
"Definition" Sea $Q = (q_{ij})$ sea la matrix de transicion de una cadena de Markov, Suponiendo que es $S=(S_1,\dots,S_M)$ con $S_i\geq0, \sum_{i}S_i = 1$ tal que $$S_{i}q_{ij}=S_{j}q_{ji}$$ Para todos los estados $i$ and $j$. Esta ecuacion es llamada $reversibilidad$ o \textit{Detalle de la condición de equilibrio}, y nosotros diremos que la cadena \textit{reversible} respecto a $S$ si la contiene.

Dada una matriz de transición, si podemos encontrar un vector no negativo $s$ cuyas componentes suman 1 y que satisfaga la condición de reversibilidad, entonces $s$ es automáticamente una distribución estacionaria.